\documentclass[a4paper]{article}
\usepackage[utf8]{inputenc}
\usepackage[frenchb]{babel}
\usepackage[T1]{fontenc}
\usepackage{fullpage}
\usepackage{graphicx}
\usepackage{listings}

\title{Projet multicore programming}
\author{Thibault \bsc{Bèziers la Fosse}, Benjamin \bsc{Moreau}}
\date{28 Février 2016}

\begin{document}

\maketitle

\section{Introduction}
    \paragraph{}
    L'objectif de ce projet est d'implémenter une version parallélisée avec le langage \emph{C++} de l'algorithme de branch \& bound par intervalles afin d'encadrer le minimum d'une fonction de deux variables réelles.
    
    La parallélisation du code se fera en deux étapes. Dans un premier temps, nous utiliserons \emph{MPI} pour effectuer le calcul en coopération sur plusieurs machines. Dans un second temps, nous paralléliserons le code au sein de chaque machine à l'aide de \emph{OpenMP}. Dans chaque partie, nous justifierons nos choix d'implémentation et présenterons les résultats obtenus.
    
\section{Parallélisation avec \emph{MPI}}
    \subsection{Implémentation}
    \paragraph{}
    A l'aide de \emph{MPI}, nous avons décidé de paralléliser le programme au début de l’exécution. La boite contenant la fonction est coupée en $N$ sous-boites chacune prise en charges par les $N$ processus.
    Le premier processus demande à l'utilisateur la fonction à analyser puis \emph{broadcast} le choix à tout les autres. Il demande ensuite la précision à l'utilisateur et la \emph{broadcast} aux autres processus.
    Enfin, chaque processus appelle l'algorithme de \emph{Branch \& Bound} sur sa sous-boite calculée à l'aide du nombre de processus $N$ et de son rang $K$. Le minimum est trouvé en faisant un \emph{reduce} des minimums trouvés par chaque processus.
    
    \paragraph{}
    L’inconvénient de notre solution est qu'elle ne prend pas en compte la structure \emph{minimizer}. Pour régler ce problème, il faudrait que cette dernière soit partagée entre les processus \emph{MPI}. Chaque processus travail donc de façon indépendante sur son sous-espace.
    \subsection{résultats}
    \paragraph{}
    
\section{Parallélisme à mémoire partagée avec \emph{OpenMP}}
	\paragraph{}
	Nous avons choisis \emph{OpenMP} pour la parallélisation à mémoire partagée car il très facile d'utilisation. De plus, l'utilisation de \emph{pragma} permet à l'utilisateur d'utiliser le code source sans forcement posséder la librairie \emph{OpenMP}. Malgré sa souplesse d'utilisation, il offre de bonnes performances.
    \subsection{Implémentation}
    \paragraph{}
    La parallélisation du code ce fait au sein de l'algorithme de \emph{Branch \& Bound}. Dans cette fonction, si la précision n'est pas suffisante, l'espace cubique contenant la fonction est coupé en 4 sous-espaces et la fonction est rappelée récursivement sur ces sous-boites. Nous parallélisons donc l'appelle récursif de cette fonction. A chaque appelle de la fonction, 4 exécutions parallèles de code sont donc lancés.
    
    Ces exécutions parallèles accèdent à une variable commune qu'il faut donc protéger : la liste de minimum courant. Nous utilisons donc un \emph{pragma} définissant une section critique à chaque fois que la variable est modifiée au sein de la fonction parallélisée.
    \subsection{résultats}
    \paragraph{}
    
\section{Conclusion}
    \paragraph{}
    
    

\end{document}